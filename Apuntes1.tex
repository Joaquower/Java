%\documentclass[12pt,a4paper]{article} el tamaño normala en muchos lados
\documentclass[letterpaper]{article} %  El tamaño de una hoja carta
\usepackage[spanish]{babel}
\usepackage[utf8]{inputenc}
\usepackage{graphicx}

\title{Mi ensayo}
\author{Joaquin}
\date{\today}

\begin{document}
\maketitle
\section*{Introduccion al LAT3X}
\leftskip=15pt %sangria izquierda
\rightskipp=0pt %sangria derecha
\parindent=0pt %margenes de la sangria
Hata aqui he aprendido que latex se maneja por secciones y subsecciones para colocar las cosas, igual hay que manejar muy bien los paquetes ya que se usan para todo, todo latex podria decirse que es la manipulacion de paquetes.

De igual forma aprendi que despues de los paquetes se suele poner el titulo, auto y fecha del proyecto y esos datos los acomoda en la portada. El proyecto se inicia con un begin y se termina con un end.

En VSC se utiliza ALT + Z para hacer que se acomode todo un texto en varias lineas para que lo puedas visualizar todo en vez de que se recorr a la derecha y se haga una super linea.

De igual forma aprendi el uso de imagenes con la funcion figure, se pueden centrar y ajustar su tamaño y angulo.

Tambien aprendi a crear listas no ordenadas con la funcion itemize.

\subsection*{Subseccion 1}
Hola este es el cuerpo de la primera subseccion.
\begin{figure}[h]
    \centering
    \includegraphics*[with=0.5\textwidth,height=10,angle=0]{imagefile}
    \caption{Descripcion de la imagen}
\end{figure}
\begin{itemize}
    \item Elemento1
    \item Elemento2
    \item Elemento3
    \item Elemento4
\end{itemize}
\end{document}